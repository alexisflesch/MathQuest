
\begin{clickers}
    À quelle fonction usuelle l'expression ci-dessous correspond-elle?
    \[ x-\frac{x^3}{3!}+\frac{x^5}{5!} + x^6\varepsilon(x), \quad \varepsilon(x)\underset{x\to0}{\rightarrow}0\]
    \begin{itemize}
        \item cosinus
        \item sinus
        \item exponentielle
        \item logarithme népérien
        \item Aucune des réponses ci-dessus
        \item Je ne sais pas
    \end{itemize}
\end{clickers}



\begin{clickers}
    Soit \(f\) une fonction telle que:
    \[ f'(x) = 2-x+3x^2+x^2\varepsilon(x), \quad \varepsilon(x)\underset{x\to0}{\rightarrow}0.\]
    Alors, le \(DL_3(0)\) de \(f\) est donné par:
    \[ f(x) = 2x-\frac{x^2}{2}+x^3+x^3\varepsilon_1(x), \quad \varepsilon_1(x)\underset{x\to0}{\rightarrow}0.\]
    \begin{itemize}
        \item Vrai
        \item Faux
        \item Je ne sais pas
    \end{itemize}
\end{clickers}


\begin{clickers}
    Quelle opération ne peut-on pas faire avec les développements limités?
    \begin{itemize}
        \item Addition
        \item Composition
        \item Intégration
        \item Soustraction
        \item Multiplication
        \item Dérivation
        \item Aucune des opérations ci-dessus n'est interdite.
    \end{itemize}
\end{clickers}


\begin{clickers}
    Soit \(f\) une fonction dont le développement limité à l'ordre \(3\) au voisinage de \(0\) est donné par:
    \[ f(x) = 1+2x + 2x^3 + x^3\varepsilon(x)\]
    Alors, la tangente à la courbe représentative \(\mathscr{C}_f\) de \(f\) au voisinage de \(0\) :
    \begin{itemize}
        \item est au-dessus de \(\mathscr{C}_f\)
        \item est en-dessous de \(\mathscr{C}_f\)
        \item traverse \(\mathscr{C}_f\)
        \item On ne peut pas répondre
        \item Je ne sais pas
    \end{itemize}
\end{clickers}


\begin{clickers}
    Sans faire le calcul du développement limité, sélectionner la seule formule exacte parmi les propositions ci-dessous:
    \begin{itemize}
        \item \(\displaystyle\cos(\sin(x)) = 1 - \frac{1}{2}x^2 +x^2\varepsilon(x)\)
        \item \(\displaystyle\cos(\sin(x)) = -\frac{1}{2}x^2 +x^2\varepsilon(x)\)
        \item \(\displaystyle\cos(\sin(x)) = x-x^2 +x^2\varepsilon(x)\)
        \item Je ne sais pas
    \end{itemize}
\end{clickers}


\begin{clickers}
    Si \(f\) admet un \(DL_2(a)\) donné par :
    \[ f(x) = a_0 + a_1 (x-a) + a_2 (x-a)^2 + (x-a)^2 \varepsilon(x), \quad \underset{x\to a}{\lim}\varepsilon(x)=0,\]
    alors \(f\) est deux fois dérivable en \(a\) et \(a_2=\dfrac{f''(a)}{2!}\).
    \begin{itemize}
        \item Vrai
        \item Faux
        \item Je ne sais pas
    \end{itemize}
\end{clickers}


\begin{clickers}
    Au voisinage de \(0\), \(o(x^2)+o(x^3)=\cdots\)
    \begin{itemize}
        \item \(o(x^2)\)
        \item \(o(x^3)\)
        \item \(o(x^5)\)
        \item \(o(x^6)\)
        \item Aucune des réponses ci-dessus
    \end{itemize}
\end{clickers}

\begin{clickers}
    Si au voisinage de \(0\), \(f(x)=1-x^2/2+x^2\varepsilon(x)\) alors, l'équation de la tangente à la courbe représentative de \(f\) au voisinage de \(0\) est:
    \begin{itemize}
        \item \(y=1-x^2/2\)
        \item \(y=1\)
        \item Aucune des réponses ci-dessus
        \item Je ne sais pas
    \end{itemize}
\end{clickers}

\begin{clickers}
    Si au voisinage de \(0\), \(f(x)=1-x^2/2+x^2\varepsilon(x)\), que peut-on dire?
    \begin{itemize}
        \item \(f\) est continue en \(0\)
        \item \(f\) est deux fois dérivable en \(0\)
        \item L'équation de la tangente en \(0\) est \(y=1-x^2/2\)
        \item La courbe de \(f\) est au-dessus de sa tangente en \(0\)
        \item Je ne sais pas
    \end{itemize}
\end{clickers}


\begin{clickers}
    Quelle formule correspond au déveoppement limité à l'ordre \(3\) de \(\cos\) en \(0\)?
    \begin{itemize}
        \item \(x-x^3/3\)
        \item \(x-x^3/6\)
        \item \(1-x^2/2\)
        \item \(1+x+x^2/2+x^3/6\)
        \item \(1+x+x^2+x^3\)
        \item Aucune des réponses ci-dessus
    \end{itemize}
\end{clickers}


\begin{clickers}
    Si \(f(x)=1+x+o(x)\), alors
    \begin{itemize}
        \item \(f(2x) = 1+x+o(x^2)\)
        \item \(f(x)^2 = 1 + x^2+o(x^2)\)
        \item \(2f(x) = 1+2x+o(x)\)
        \item \(f(x^2) = 1+x+o(x)\)
        \item Aucune des réponses ci-dessus
    \end{itemize}
\end{clickers}


\begin{clickers}
    Soient \(f\) et \(g\) deux fonctions admettant chacune un développement limité à l'ordre \(1\) au voisinage de \(0\) et vérifiant \(f(0)=g(0)=0\). Laquelle de ces fonctions admet nécessairement une limite lorsque \(x\to 0\)?
    \begin{itemize}
        \item \(f(x)/g(x)\)
        \item \(f(x)g(x)/x^2\)
        \item \(x^2f(x)/g(x)\)
        \item Aucune des fonctions ci-dessus
    \end{itemize}
\end{clickers}


\begin{clickers}
    Le développement limité à l'ordre \(2\) au voisinage de \(0\) de \((1+x)^x\) est donné par:
    \[ (1+x)^x = 1+x^2 + \frac{x(x-1)}{2}x^2+o(x^2)\]
    \begin{itemize}
        \item Vrai
        \item Faux
        \item Je ne sais pas
    \end{itemize}
\end{clickers}

\begin{clickers}
    Sans faire le calcul du développement limité, sélectionner la seule formule correcte:
    \begin{itemize}
        \item \( \mathrm{arctan}(x) = x-x^3/3 + o(x^3)\)
        \item \( \mathrm{arctan}(x) = 1 + x - x^3/3 + o(x^3)\)
        \item \( \mathrm{arctan}(x) = x-x^2 - x^3/3 + o(x^3)\)
    \end{itemize}
\end{clickers}
